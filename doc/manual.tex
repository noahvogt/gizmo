\documentclass{article}


\title{Gizmo Manual}
\begin{document}
\maketitle
\tableofcontents

\section*{The Gizmo Notation}
Gizmo uses a simple notation which only describes pitch and length. The notation is separated in two  parts by a comma  (``,''). The first part represents the note length as a three digit float while 1.00 represents a quarter note. The second part represents the pitch. It is described with a capital letter and a integer for the octave. If an accidental is needed it is written after the letter by a ``\# '' sign ($\sharp$) or a ``$-$'' sign ($\flat$).

%here an example for the gizmo notation ======================

\section{The Motive Generator}
The motiv generator invents a new motive. To use it just execute the motiv-gen file. The command can take some arguments:
$$\rm{motive-gen} <stout\, ||\, file>\,<length>\,<velocity>\,<firstNote>$$

\begin{itemize}

\item The first argument decides if the output is juste on the console or if it  writes a new file \footnote {The file is always named ``gen-motiv'' and will overwrite itself if it already exists} containing the music in gizmo notation.

\item The second argument determines the overall length of the motive mesured in quarter notes. Only positiv intergers are allowd.

\item The third argument is 0, 1 or 2 and is a way to manipulate the velocity. 0 will produce a slow motive while 2 will produce a fast motive.

\item With the last argument it is possible to specifiy the first note which will be used in the motive by using the gizmo notation.

\end{itemize}

Here an example with all arguments used:
\begin{center}
 motiv-gen file 4 2 D\# 4
\end{center}
 It is necessary to use all previous arguments of the choosen argumen. However, it is possible to input a ``wrong'' argument (like -1) to default any argument. So for example:
 \begin{center}
 motiv-gen -1 4 -1 F-4
 \end{center}
 would yield a motiv which begins with f-flat and has length of a whole note.

\section{Theme Generator}
%================================
\section{Theme }



\end{document}
